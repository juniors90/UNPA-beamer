\begin{frame}
    \frametitle{Tests De Integración}

    \begin{block}{Prueba De Integración Descendente}

        Las pruebas se llevan a cabo de arriba a abajo, siguiendo el 
        flujo de control o la estructura arquitectónica (por ejemplo, 
        comenzando desde la GUI o el menú principal). Los componentes
         o sistemas se sustituyen por stubs.
        
        \textbf{Ventaja:} El producto probado es muy consistente porque 
        la prueba de integración se realiza básicamente en un entorno 
        casi similar al de la realidad.
        
        Los códigos auxiliares se pueden escribir en menos tiempo porque
        en comparación con los controladores, los códigos auxiliares son 
        más sencillos de crear.
        
        \textbf{Desventajas:} La funcionalidad básica se prueba al final 
        del ciclo.
    \end{block}
\end{frame}

