\begin{frame}
    \frametitle{Tests de Regresión}
    A continuación, listamos 6 aspectos a tener en cuenta:
    
    \begin{enumerate}
        \item Planificación previa,
        \item Uso de técnicas de diseño de casos de prueba, 
        \item Clasificación adecuada mediante suites, 
        \item Programación de mantenimientos,
        \item Definición del tipo de Regresión a realizar,
        \item Automatización.
    \end{enumerate}

    \begin{block}{Planificación previa}
        Desde el inicio de la estimación de casos de prueba, es aconsejable 
        \textbf{asignar prioridades} de ejecución: Alta, Media, Baja y palabras
        clave como Regresión, Sanidad, etcétera para los casos de prueba.
        De esta manera se puede hacer un filtro rápido por prioridades y etiquetas,
        que permitan diferenciar rápidamente cuáles deberían ser los casos más
        importantes.
    \end{block}
\end{frame}

\begin{frame}
    \frametitle{Tests de Regresión}
    \begin{block}{Uso de técnicas de diseño de casos de prueba}
        Reconocidas entidades internacionales\footnote{International
        Software Testing Qualifications Boardy American Society for
        Quality}, mencionan varias técnicas para la creación de casos
        de prueba que permiten una mejora, de tal manera que al
        aplicarlas se obtienen menos casos de prueba y un mayor
        factor de \textbf{coverage} o bien una base de creación empírica.

        Al tener un menor número de casos de prueba, se reducen las 
        pruebas exhaustivas y tiempos de ejecución.
    \end{block}
\end{frame}

\begin{frame}
    \frametitle{Tests de Regresión}
    \begin{block}{Clasificación adecuada mediante suites}
        La división de \textbf{casos de prueba} según los módulos
        de un sistema, facilitará la selección de casos al momento
        de construir un \textbf{Plan de Pruebas} para Regresión.
        
        En las carpetas o suites deberían evitarse nombres ambiguos o 
        diferentes a lo definido en el requerimiento, o en los nombres 
        que se muestran en el sistema.
    \end{block}
\end{frame}

\begin{frame}
    \frametitle{Tests de Regresión}
    \begin{block}{Programación de mantenimientos}
        Conforme nuestra aplicación crece o sufre cambios y mejoras, 
        se hace más necesario dedicar tiempo a revisar si estos 
        cambios afectan nuestros casos de pruebas actuales, invalidándolos o 
        requiriendo actualización. Si sabemos que el cambio los afectará,
        debemos incluir tareas de actualización.
         
        Si estamos bajo un modelo de un sistema antiguo con casos
        que nunca han tenido mantenimiento, lo recomendable es incluir 
        tareas periódicas para la revisión de todos los casos existentes.
    \end{block}
\end{frame}

\begin{frame}
    \frametitle{Tests de Regresión}
    \begin{block}{Definición del tipo de Regresión a realizar}
        No todas las pruebas de regresión implican una ejecución 
        del $100\%$ de los casos de prueba (Full Regression).
        
        En muchos casos, los cambios realizados afectan componentes 
        específicos por lo que la regresión podría centrarse en 
        esos módulos y la verificación del resto del sistema podría 
        realizarse con una prueba de humo.
        
        En estos casos es necesario un análisis de dependencias 
        para tener certeza de que los cambios por aplicar efectivamente 
        no afectan otras partes no contempladas en la regresión del 
        componente.
    \end{block}
\end{frame}

\begin{frame}
    \frametitle{Tests de Regresión}
    \begin{block}{Automatización}
        Las pruebas de regresión suelen ser procesos largos y 
        al incorporar \textbf{configuracionde para las pruebas de automatización}
        se consiguen varias ventajas.

        \begin{itemize}
            \item La capacidad de ejecución aumenta, al 
            tiempo que la duración de las pruebas se reduce.
            
            \item Los scripts pueden ejecutarse tantas veces como se 
            requiera sin que esto implique desgaste en el equipo.
            
            \item Se pueden incorporar diferentes entornos en
            la prueba sin que esto aumente el tiempo de las mismas, 
            pues pueden ejecutarse en paralelo.
        \end{itemize}
    \end{block}
\end{frame}

\begin{frame}
    \frametitle{Bibliografía}
   
    %Algunos libros que recomiendo: 
    \nocite{wong1997study, yoo2012regression, leung1990study, leung1989insights}
   
    \bibliographystyle{plain}
    \bibliography{biblio}
  \end{frame}