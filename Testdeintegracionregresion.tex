\documentclass[xetex, 11pt,spanish]{beamer}
\usepackage[spanish]{babel}
\usetheme{Madrid}

\makeatletter
\def\th@mystyle{%
	\normalfont % body font
	\setbeamercolor{block title example}{bg=orange,fg=white}
	\setbeamercolor{block body example}{bg=orange!20,fg=black}
	\def\inserttheoremblockenv{exampleblock}
}
\makeatother
\theoremstyle{mystyle}
\newtheorem*{remark}{Remark}

\usepackage{fontspec}
\usepackage{unicode-math}

\begin{document}
\author{Ferreira Juan David}
	\title{Tests De Integración/Regresión}
	\subtitle{}
	\logo{}
	\institute{}
	\date{}
	%\subject{}
	%\setbeamercovered{transparent}
	%\setbeamertemplate{navigation symbols}{}
	\begin{frame}[plain]
		\maketitle
	\end{frame}
	\begin{frame}
		\frametitle{Tests de Integración/Regresión}
		Las pruebas de integración dentro del software testing chequean la integración o interfaces entre componentes, interacciones con diferentes partes del sistema, como un sistema operativo, sistema de archivos y hardware o interfaces entre sistemas. Las pruebas de integración son un aspecto clave del software testing.
		
		Es esencial que un probador de software tenga una buena comprensión de los enfoques de prueba de integración, para lograr altos estándares de calidad y buenos resultados. 
	\end{frame}

	\begin{frame}
    \frametitle{Tests De Integración}

    Dentro del software testing existen muchos tipos o enfoques diferentes
    para las pruebas de integración. Los enfoques más populares y de uso
    frecuente son

    \begin{itemize}
        \item las pruebas de integración Big Bang, 
        \item las pruebas de integración descendente, 
        \item las pruebas de integración ascendente y 
        \item las pruebas de integración incremental.
    \end{itemize}
    
    La elección del enfoque depende de varios factores como el costo, la 
    complejidad, la criticidad de la aplicación, etc.

\end{frame}

\begin{frame}
    \frametitle{Tests De Integración}

    Hay muchos tipos menos conocidos de pruebas de integración, como la 
    integración las siguientes:

    \begin{itemize}
        \item las pruebas de integración de servicios distribuidos,
        \item las pruebas de integración sándwich, 
        \item las pruebas de integración de la red troncal,
        \item las pruebas de integración de alta frecuencia,
        \item las pruebas de integración de integración de capas, etc.
    \end{itemize}
    
\end{frame}


	\begin{frame}
    \frametitle{Tests De Integración}

    \begin{block}{Prueba De Integración Big Bang}
        En las pruebas de integración de Big Bang, todos los componentes o 
        módulos se integran simultáneamente, después de lo cual todo se 
        prueba como un todo.
        
        \textbf{Ventaja:} Las pruebas de Big Bang tienen la ventaja de que 
        todo está terminado antes de que comiencen las pruebas de integración.
        
        \textbf{Desventaja:} Podemos detectar los defectos de la interfaz 
        clave al final del ciclo.

        Es necesario crear los controladores de prueba para los módulos 
        en todos los niveles excepto el control superior.
    \end{block}
\end{frame}


	\begin{frame}
    \frametitle{Tests De Integración}

    \begin{block}{Prueba De Integración Descendente}

        Las pruebas se llevan a cabo de arriba a abajo, siguiendo el 
        flujo de control o la estructura arquitectónica (por ejemplo, 
        comenzando desde la GUI o el menú principal). Los componentes
         o sistemas se sustituyen por stubs.
        
        \textbf{Ventaja:} El producto probado es muy consistente porque 
        la prueba de integración se realiza básicamente en un entorno 
        casi similar al de la realidad.
        
        Los códigos auxiliares se pueden escribir en menos tiempo porque
        en comparación con los controladores, los códigos auxiliares son 
        más sencillos de crear.
        
        \textbf{Desventajas:} La funcionalidad básica se prueba al final 
        del ciclo.
    \end{block}
\end{frame}


	\begin{frame}
    \frametitle{Tests De Integración}

    \begin{block}{Prueba De Integración Descendente}

        Las pruebas se llevan a cabo desde la parte inferior del flujo de
        control hacia arriba. Los componentes o sistemas se sustituyen 
        por controladores.
        
        \textbf{Ventaja:} En este enfoque, el desarrollo y las pruebas 
        se pueden realizar juntos para que el producto o la aplicación 
        sea eficiente y de acuerdo con las especificaciones del cliente.
        
        Los códigos auxiliares se pueden escribir en menos tiempo porque
        en comparación con los controladores, los códigos auxiliares son 
        más sencillos de crear.
        
        \textbf{Desventajas:} La funcionalidad básica se prueba al final 
        del ciclo.
    \end{block}
\end{frame}


	\begin{frame}
    \frametitle{Tests De Integración}

    \begin{block}{Prueba De Integración Incremental}

        Otro enfoque es que todos los programadores se integran uno 
        por uno y se realiza una prueba después de cada paso.
        
        \textbf{Ventaja:} los defectos se encuentran temprano en un 
        ensamblaje más pequeño cuando es relativamente fácil 
        detectar la causa.
        
        Dentro de las pruebas de integración incremental existe una gama de posibilidades, en parte dependiendo de la arquitectura del sistema.
        
        \textbf{Desventajas:} Una desventaja es que puede llevar mucho 
        tiempo ya que los stubs y los controladores deben desarrollarse 
        y usarse en la prueba.
    \end{block}
\end{frame}



	\begin{frame}
    \frametitle{Tests de Regresión}
    \begin{block}{Prueba De Regresión}
        Consiste en probar un sistema que ha sido  
        analizado previamente para asegurar que no se haya introducido 
        algún tipo de  defecto  como resultado de cambios realizados.
\end{block}


La planificación de estas pruebas no es tan sencilla, especialmente
cuando el tiempo y los recursos son limitados. Como consecuencia 
de esa limitación, se genera presión y  los planes de pruebas suelen 
acortarse o se pierde tiempo en el ajuste ocasionando un aumento en 
la probabilidad de fallo humano al no seleccionar casos más adecuados.
%    A continuación, le detallamos los 6 aspectos indispensables que 
%    puede tener en cuenta:
\end{frame}
	\begin{frame}
    \frametitle{Tests de Regresión}
    A continuación, listamos 6 aspectos a tener en cuenta:
    
    \begin{enumerate}
        \item Planificación previa,
        \item Uso de técnicas de diseño de casos de prueba, 
        \item Clasificación adecuada mediante suites, 
        \item Programación de mantenimientos,
        \item Definición del tipo de Regresión a realizar,
        \item Automatización.
    \end{enumerate}

    \begin{block}{Planificación previa}
        Desde el inicio de la estimación de casos de prueba, es aconsejable 
        \textbf{asignar prioridades} de ejecución: Alta, Media, Baja y palabras
        clave como Regresión, Sanidad, etcétera para los casos de prueba.
        De esta manera se puede hacer un filtro rápido por prioridades y etiquetas,
        que permitan diferenciar rápidamente cuáles deberían ser los casos más
        importantes.
    \end{block}
\end{frame}

\begin{frame}
    \frametitle{Tests de Regresión}
    \begin{block}{Uso de técnicas de diseño de casos de prueba}
        Reconocidas entidades internacionales\footnote{International
        Software Testing Qualifications Boardy American Society for
        Quality}, mencionan varias técnicas para la creación de casos
        de prueba que permiten una mejora, de tal manera que al
        aplicarlas se obtienen menos casos de prueba y un mayor
        factor de \textbf{coverage} o bien una base de creación empírica.

        Al tener un menor número de casos de prueba, se reducen las 
        pruebas exhaustivas y tiempos de ejecución.
    \end{block}
\end{frame}

\begin{frame}
    \frametitle{Tests de Regresión}
    \begin{block}{Clasificación adecuada mediante suites}
        La división de \textbf{casos de prueba} según los módulos
        de un sistema, facilitará la selección de casos al momento
        de construir un \textbf{Plan de Pruebas} para Regresión.
        
        En las carpetas o suites deberían evitarse nombres ambiguos o 
        diferentes a lo definido en el requerimiento, o en los nombres 
        que se muestran en el sistema.
    \end{block}
\end{frame}

\begin{frame}
    \frametitle{Tests de Regresión}
    \begin{block}{Programación de mantenimientos}
        Conforme nuestra aplicación crece o sufre cambios y mejoras, 
        se hace más necesario dedicar tiempo a revisar si estos 
        cambios afectan nuestros casos de pruebas actuales, invalidándolos o 
        requiriendo actualización. Si sabemos que el cambio los afectará,
        debemos incluir tareas de actualización.
         
        Si estamos bajo un modelo de un sistema antiguo con casos
        que nunca han tenido mantenimiento, lo recomendable es incluir 
        tareas periódicas para la revisión de todos los casos existentes.
    \end{block}
\end{frame}

\begin{frame}
    \frametitle{Tests de Regresión}
    \begin{block}{Definición del tipo de Regresión a realizar}
        No todas las pruebas de regresión implican una ejecución 
        del $100\%$ de los casos de prueba (Full Regression).
        
        En muchos casos, los cambios realizados afectan componentes 
        específicos por lo que la regresión podría centrarse en 
        esos módulos y la verificación del resto del sistema podría 
        realizarse con una prueba de humo.
        
        En estos casos es necesario un análisis de dependencias 
        para tener certeza de que los cambios por aplicar efectivamente 
        no afectan otras partes no contempladas en la regresión del 
        componente.
    \end{block}
\end{frame}

\begin{frame}
    \frametitle{Tests de Regresión}
    \begin{block}{Automatización}
        Las pruebas de regresión suelen ser procesos largos y 
        al incorporar \textbf{configuracionde para las pruebas de automatización}
        se consiguen varias ventajas.

        \begin{itemize}
            \item La capacidad de ejecución aumenta, al 
            tiempo que la duración de las pruebas se reduce.
            
            \item Los scripts pueden ejecutarse tantas veces como se 
            requiera sin que esto implique desgaste en el equipo.
            
            \item Se pueden incorporar diferentes entornos en
            la prueba sin que esto aumente el tiempo de las mismas, 
            pues pueden ejecutarse en paralelo.
        \end{itemize}
    \end{block}
\end{frame}

\begin{frame}
    \frametitle{Bibliografía}
   
    %Algunos libros que recomiendo: 
    \nocite{wong1997study, yoo2012regression, leung1990study, leung1989insights}
   
    \bibliographystyle{plain}
    \bibliography{biblio}
  \end{frame}
\end{document}
